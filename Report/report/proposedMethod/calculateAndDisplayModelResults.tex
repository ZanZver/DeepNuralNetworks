\subsection{Calculate and display model results}
To calculate and display results, we first need to grab the data. We do this by getting history data that we have created from above.
\begin{lstlisting}[language=Python, caption=Create dictionary of history values]
history_dict = history.history
history_dict.keys()
\end{lstlisting}
This step is not necessary, but it is convenient. What we do is, extract values based on keys and save it to local variables.
\begin{lstlisting}[language=Python, caption=Save data from dict to local variables]
training_acc = history_dict['accuracy']
val_acc = history_dict['val_accuracy']

training_loss = history_dict["loss"]
val_loss = history_dict["val_loss"]
\end{lstlisting}
In the next step, we plot our first graph as training and validation accuracy.
\begin{lstlisting}[language=Python, caption=Plot training and validation accuracy]
epochs = range(1, len(val_acc) + 1)

plt.plot(epochs, training_acc, 'bo', label="Training Accuracy")
plt.plot(epochs, val_acc, 'b', label='Validation Accuracy')

plt.title('Plot training and validation accuracy')

plt.legend()
plt.show()
\end{lstlisting}
Finaly, we plot the last graph as training and validation loss.
\begin{lstlisting}[language=Python, caption=Plot training and validation loss]
epochs = range(1, len(training_loss) + 1)

plt.plot(epochs, training_loss, 'bo', label='Training Loss')
plt.plot(epochs, val_loss, 'b', label='Validation Loss')

plt.title('Plot training and validation loss')

plt.legend()
plt.show()
\end{lstlisting}